\subsection{线性回归\&逻辑回归}
\textbf{\checkmark 2020-09-30}\\
线性回归研究的问题是多个变量中某个变量和其他变量之间存在的线性关系,相当于用多个变量线性表示某个变量,某个变量就称为因变量,其他变量就称为自变量。用数学语言来描述的话就是这样的:
$$
y=b + \sum_{i=1}^{n} w_i \cdot x_i 
$$
在n等于1时,相当于根据数据拟合一条直线;在n大于1的时候,就是多元线性回归了,此时拟合一个平面。自变量也可以称为特征。
构建线性回归模型时,重要的有这几个点:发现相关性较高的特征,发现与因变量无关的特征,得到最后实际使用的n个特征,对于实值特征的范围进行约束,对于类别特征的类别进行处理。其实,以上这些主要是针对数据的初步分析和预处理。
得到处理后的数据,接下来可以通过随机梯度下降的方法得到最优的权重。在线性回归中常使用的目标函数是均方误差函数。为了避免模型过拟合,目标函数中还可以加入正则化项,对特征权重进行限制。

其实线性回归模型很像神经络中的一部分———一个激活函数为恒等映射的神经元,也可以看做一个神经网络模型———一个只有一层的神经网络模型。

\textbf{逻辑回归(Logistic regression, LR)},也是在线性回归的基础上的一个分类模型。如果把线性回归看做神经网络单元,那么把激活函数替换为非线性的激活函数就是LR了。从数学形式来看,LR是这样的:
$$
z = b + \sum_{i=1}^{n} w_i \cdot x_i 
$$
$$
\bar{y} = \frac{1}{1 + e^{-z}} = \frac{e^z}{1 + e^z}
$$
针对一个样本$x$,LR模型得到的就是$\bar{y}$,很明显这是一个0到1之间的值,即一个概率值,当对样本进行类别划分后(划分为0、1),$\bar{y}$是类别为1的概率。二分类LR的目标函数定义如下:
$$
L(w_1,...,w_n) = \prod_{i = 1}^{m} (\bar{y^i} )^{y^i} \cdot ( 1 - \bar{y^i}) ^ {1 - y^i}
$$
其中 $y^i$为样本$\boldsymbol{x^i}$的实际类别,$y^i \in \{0, 1\}$。显然这是一个关于权重和偏置的最大似然函数,对其取对数后得到:
$$
loss = log L = \sum_{i=1}^{m} \left( y^i log (\bar{y^i}) + (1 - y^i) log ( 1 - \bar{y^i}) \right)
$$

很明显,线性回归和逻辑回归的目标函数是不一样的,那么为什么LR不适用均方误差损失函数呢?理论上LR也是可以使用均方误差损失函数的,但是均方误差损失在进行SGD时存在一个问题,当预测值与真实值相差越大时,参数变化的越小,训练的越慢({\color{red}这个可以通过均方误差损失函数进行SGD时对参数的梯度可以看出})。上述的对数似然函数也可以叫做交叉熵。

对于多类别(例如类别数为$\mathcal{L}$)的LR,可以训练$\mathcal{L}$个二分类的LR,每个LR只输出样本属于某个类别的概率,最后进行集成得到最终的输出结果。此时的目标函数则会发生一点变化:
$$
loss = \sum_{i=1}^{m} \sum_{l=1}^{\mathcal{L}} \mathbbm{1}\{y^i = l\} log \frac{e^{ \boldsymbol{w^l} \cdot \boldsymbol{x^i} } }{ \sum_{j=1}^{\mathbb{L}} e^{ \boldsymbol{w^j} \cdot \boldsymbol{x^i}} }
$$
其中$\boldsymbol{w^l}$是第$l$个LR模型的权重向量(包括了偏置)。

\textbf{\checkmark 2021-10-14}\\
使用最小二乘法(使用均方误差来求解模型)求解线性回归时,令$\boldsymbol{X}, \boldsymbol{w}$分别表示数据集和权重向量(包含了偏置),则均方误差:
$$
E_{\boldsymbol{\hat w}} = (\boldsymbol{y} - \boldsymbol{X} \boldsymbol{\hat w})^T (\boldsymbol{y} - \boldsymbol{X} \boldsymbol{\hat w})
$$

$E_{\boldsymbol{\hat w}}$对$\boldsymbol{\hat w}$求导:
$$
\frac{\partial E_{\boldsymbol{\hat w}}}{\partial \boldsymbol{\hat w}} = 2 \boldsymbol{X}^T(\boldsymbol{X}\boldsymbol{\hat w} - \boldsymbol{y})
$$

如果$\boldsymbol{X^T X}$是满秩矩阵或正定矩阵(可逆,正定矩阵一定满秩,因为其特征值都是正的)时,即可得到$\boldsymbol{\hat w}$的唯一解;否则,则可能存在多个解使均方误差最小,此时可根据的算法的归纳偏好(学习算法对某种类型的假设的偏好,什么样的模型更好,例如拟合曲线时,存在多条曲线符合要求,但是一般选择更平滑的)决定,如引入正则化。

\textbf{\checkmark 2021-10-16}\\
\textbf{从广义线性模型到逻辑回归} \\
对于样例$(\boldsymbol{x}, y))$,当我们用线性模型的预测值逼近$y$时,就得到了线性回归模型,也可以用用线性模型来逼近$y$的衍生物,该衍生物是$y$的函数,即:
$$
g(y) = \boldsymbol{w}^T \boldsymbol{x} + b
$$
则有:
$$
y = g^{-1}(\boldsymbol{w}^T \boldsymbol{x} + b)
$$
此时,称$g(\cdot)$为联系函数。当基于线性模型做分类时,只需要找到一个联系函数,将线性回归模型的预测值(\textbf{注意线性回归的预测目标不是类别,但是要将这个不是类别的值与类别概率联系起来})与分类任务的真实标记$y$联系起来,即找到$g^{-1}$。在逻辑回归中,令$g^{-1}(z) = \frac{1}{1 + e ^{-z}}$,即为对数几率函数,则:
$$
y = \frac{1}{1 + e^{-(\boldsymbol{w}^T \boldsymbol{x} + b)}}
$$
其实,根据上式可以反推出$g$来,即:
$$
g(y) = \ln \frac{y}{1 - y} = \boldsymbol{w}^T \boldsymbol{x} + b
$$
其中$g(y) = \ln \frac{y}{1 - y}$就是\textbf{对数几率}。所以,可以这样描述逻辑回归:\textbf{用线性回归来拟合真实标记的对数几率}。

\textbf{\checkmark 2021-10-28}\\
\textbf{线性回归与均方误差}\ \ \ \
由于真实数据通常是存在噪声的,所以预测值一般是约等于真实值的,即$\hat{y} = \boldsymbol{w}^T \boldsymbol{x} \approx y$。用$\epsilon \sim \mathcal{N}(0, \sigma^2)$表示噪声,则有:
\begin{align}
	y &= \hat{y} + \epsilon \nonumber \\
	  &= \boldsymbol{w}^T \boldsymbol{x} + \epsilon 	\nonumber
\end{align}
假设对于样本$(\boldsymbol{x}, y)$,$P(y | \boldsymbol{x}, \boldsymbol{w}, \epsilon) \sim \mathcal{N}(\boldsymbol{w}^T \boldsymbol{x}, \sigma^2)$,即$P(y | \boldsymbol{x}, \boldsymbol{w}, \epsilon) = \frac{1}{\sqrt{2\pi} \sigma} e^{-\frac{(y - \boldsymbol{w}^T \boldsymbol{x})^2}{2 \sigma^2}}$。目标是求$\boldsymbol{w}$使$P(y | \boldsymbol{x}, \boldsymbol{w}, \epsilon)$最大,通过极大似然可得:
\begin{align}
	\boldsymbol{w} &= \mathop{argmax}_{\boldsymbol{w}} P(D | \boldsymbol{x}, \boldsymbol{w}, \epsilon) \nonumber \\
		&= \mathop{argmax}_{\boldsymbol{w}} \prod_{i=1}^N e^{-\frac{(y - \boldsymbol{w}^T \boldsymbol{x})^2}{2 \sigma^2}} \nonumber \nonumber \\
		&= \mathop{argmax}_{\boldsymbol{w}} \prod_{i=1}^N -\frac{(y_i - \boldsymbol{w}^T \boldsymbol{x})^2}{2 \sigma^2} \nonumber \\
		&= \mathop{argmin}_{\boldsymbol{w}} (y_i - \boldsymbol{w}^T \boldsymbol{x}) \nonumber \\
		&= \mathop{argmin}_{\boldsymbol{w}} \sum_{i=1}^N (y_i - \boldsymbol{w}^T \boldsymbol{x})^2  \nonumber
\end{align}
关于这个假设,可以这样理解:当给定$\boldsymbol{x}, \boldsymbol{w}, \epsilon$时,真实值在$\hat{y}$附近摆动(即均值为$\hat{y}$),同时由于存在噪声,方差与$\epsilon$的方差相同。

\subsection{LDA}
\textbf{\checkmark 2020-10-16}\\
Linear Discriminant Analysis,线性判别分析。一种经典的线性学习方法,因为最早由Fisher提出,故也称为\textit{Fisher 判别分析}。

\paragraph{思想}给定训练集,\textbf{设法}将样本点投影到一条直线($\boldsymbol{w}$)上,使同类别的投影到尽可能接近、异类样本点投影到尽可能原理(对比学习)。对新样本点进行分类时,将其投影到这条直线上,再根据投影点的位置来确定新样本的类别(\tbc{red}{HOW?}),或者用于数据降维·。

\paragraph{两个类别的LDA}给定数据集$D = \{(x_i, y_i)\}_{i=1}^{m}$,其中$x_i \in \mathbb{R}^n, y_i \in \{0, 1\}$,欲求解的投影直线为$\boldsymbol{w}$。声明以下变量:
\begin{itemize}
	\item $X_0, X_1$分别表示0, 1类样本集合
	\item $\boldsymbol{\mu}_i = \frac{1}{|X_i|}\sum_{x_j \in X_i} \boldsymbol{x_j}, i\in\{0, 1\}$分别表示0, 1类样本的均值
	\item $\boldsymbol{\sum}_i = \sum_{x_j \in X_i} (\boldsymbol{x_j} - \boldsymbol{\mu}) (\boldsymbol{x_j} - \boldsymbol{\mu})^T, i\in\{0, 1\}$分别表示0, 1类样本的协方差矩阵
\end{itemize}

若将样本点投影到直线上,那么两类样本的协方差分别为$\boldsymbol{w}^T \boldsymbol{\sum_i} \boldsymbol{w}$(先计算样本点投影到直线上的值,在分类计算投影点的均值,再计算协方差)。根据LDA的目标:\textbf{同类样本点投影后的方差尽量小,异类样本点投影后距离尽量大}。注意到LDA是线性的,异类样本之间的距离可以通过类别中心投影后的距离来反映,即$||\boldsymbol{w}^T\boldsymbol{\mu}_0 - \boldsymbol{w}^T\boldsymbol{\mu}_1||_2^2$,那么LDA的目标就成了:
\begin{align}
	J(\boldsymbol{w}) &= \frac{||\boldsymbol{w}^T\boldsymbol{\mu}_0 - \boldsymbol{w}^T\boldsymbol{\mu}_1||_2^2}{\boldsymbol{w}^T \boldsymbol{\sum_0} \boldsymbol{w} + \boldsymbol{w}^T \boldsymbol{\sum_1} \boldsymbol{w}} \nonumber \\
	&= \frac{\left\|\left(\boldsymbol{w}^{\mathrm{T}} \boldsymbol{\mu}_{0}-\boldsymbol{w}^{\mathrm{T}} \boldsymbol{\mu}_{1}\right)^{\mathrm{T}}\right\|_{2}^{2}}{\boldsymbol{w}^{\mathrm{T}}\left(\boldsymbol{\Sigma}_{0}+\boldsymbol{\Sigma}_{1}\right) \boldsymbol{w}} \nonumber
\end{align}
定义类内散度矩阵(withi-class scatter matrix)$\boldsymbol{S}_w = \boldsymbol{\sum}_0 + \boldsymbol{\sum}_1$,类间散度矩阵(between-class scatter matrix)$\boldsymbol{S}_b = (\boldsymbol{\mu}_0 - \boldsymbol{\mu}_1) (\boldsymbol{\mu}_0 - \boldsymbol{\mu}_1)^T$。则$J(\boldsymbol{w})$可写成:
$$
J(\boldsymbol{w}) = \frac{\boldsymbol{w}^{\mathrm{T}} \mathbf{S}_{b} \boldsymbol{w}}{\boldsymbol{w}^{\mathrm{T}} \mathbf{S}_{w} \boldsymbol{w}}
$$
其中,分子分母都是关于$\boldsymbol{w}$的二次项,$J(\boldsymbol{w})$是关于$\boldsymbol{S}_b, \boldsymbol{S}_w, \boldsymbol{w}$的广义瑞利商。因此$J(\boldsymbol{w})$的解与$\boldsymbol{w}$的长度无关(\tbc{red}{参考瑞利商和广义瑞利商}),故可令$\boldsymbol{w}^{\mathrm{T}} \mathbf{S}_{w} \boldsymbol{w} = 1$,则新的目标为:
\begin{align}
	& \mathop{min} \limits_{\boldsymbol{w}} -\boldsymbol{w}^T\boldsymbol{S}_b\boldsymbol{w} \nonumber\\
	& s.t. \boldsymbol{w}^{\mathrm{T}} \mathbf{S}_{w} \boldsymbol{w} = 1 \nonumber
\end{align}
通过拉格朗日求解,并对$\boldsymbol{w}$求导并令之等于0可得:$\mathbf{S}_{b} \boldsymbol{w}=\lambda \mathbf{S}_{w} \boldsymbol{w}$。可以从两个角度来求解:
\begin{itemize}
	\item 两边左乘$\boldsymbol{S}_w^{-1}$可以得到:$\boldsymbol{S}_w^{-1} \mathbf{S}_{b} \boldsymbol{w}=\lambda  \boldsymbol{w}$,即$\boldsymbol{w}$为$\boldsymbol{S}_w^{-1} \mathbf{S}_{b}$的特征向量(广义特征向量)。可以解出$\boldsymbol{S}_w^{-1} \mathbf{S}_{b}$最大的$k$个特征值对于的特征向量,组成一个矩阵$\boldsymbol{W}$,来对$\boldsymbol{x}_i$降维:$\boldsymbol{W}^T \boldsymbol{x}_i$
	
	\item 从另一个角度看,$\mathbf{S}_{b} \boldsymbol{w} = (\boldsymbol{\mu}_0 - \boldsymbol{\mu}_1) (\boldsymbol{\mu}_0 - \boldsymbol{\mu}_1)^T \boldsymbol{w}$与$(\boldsymbol{\mu}_0 - \boldsymbol{\mu}_1)$同向,则令$\mathbf{S}_{b} \boldsymbol{w} = \alpha (\boldsymbol{\mu}_0 - \boldsymbol{\mu}_1)$,则$\boldsymbol{w} = \boldsymbol{S}_w^{-1} (\boldsymbol{\mu}_0 - \boldsymbol{\mu}_1)$。考虑到数值解的稳定性,通常会对$\boldsymbol{S}_w^{-1}$进行奇异值分解。用于分类时,假设各个类别的样本数据符合高斯分布,这样利用LDA进行投影后,可以利用极大似然估计计算各个类别投影数据的均值和方差,进而得到该类别高斯分布的概率密度函数。当一个新的样本到来后,我们可以将它投影,然后将投影后的样本特征分别带入各个类别的高斯分布概率密度函数,计算它属于这个类别的概率,最大的概率对应的类别即为预测类别,参考\href{https://www.cnblogs.com/pinard/p/6244265.html}{这里}
\end{itemize}


\subsection{决策树}
\textbf{\checkmark 2020-10-22}\\
Decision Tree,是一种基本的分类与回归方法,基于树形结构来进行决策。
\paragraph{思想}数据的特征构成了一个特征空间,基于训练样本对特征空间进行划分,最终得到若干子空间(即决策树的叶子节点),每个字空间的类别或值由该空间内的样本决定。决策树的学习过程就是决策树的生成过程,当然,这个决策树是指剪枝后的决策树。

\paragraph{决策树生成}给定数据集$D = \{(x_i, y_i)\_{i=1}^{m}}$,其中$x_i \in \mathbb{R}^n$,$y_i \in C$为类别(分类)或者数值(回归),其中样本的特征集为$A$,最终的决策树为$T$。

\subparagraph{分类决策树}
\begin{myenumerate}
\item 构建根节点$T$
\item 若$D$中所有样本均为同类$c$,则$T$为单节点树,并将该结点的类别设为$c$,返回$T$;
\item 若$A = \emptyset$,或者$D$中样本在$A$中各个特征上的取值均相同,则$T$为单节点树,将$D$中样本数最大的类$c$作为$T$的类别,返回$T$;
\item 根据\textbf{特征选择}的方法,从$A$中选择一个最优的特征$A_g$作为划分特征;
\item 若以$A_g$划分的收益小于收益阈值,则$T$为单节点数,将$D$中类别数最大的类$C$作为$T$的类别;
\item 遍历$A_g$的每一个值$a_i$,按照$A_g = a_i$将$D$划分为若干子集$D_i$;
\item 分别以$D_i$作为新的训练集,$A - {A_g}$作为新的特征集,递归调用上述过程,返回的子树作为$T$的子节点;
\end{myenumerate}

\tbc{red}{注意:上述过程中针对的是离散类型的特征,对于连续类型的特征,通常是选择最优的切分点,但与离散特征不同的是,连续特征在切分之后,还可以继续使用;而离散特征通常不会多次使用($A - A_g$),因为在生成时根据离散特征的每一个值将当前节点上的样本划分成多个子集,那么子集内的样本在该离散特征上的取值是一样的。当然,这也和划分方法有关,当划分离散特征时将离散特征的取值集合划分成多个不相交的子集,那么这个离散特征也应该能多次使用;同理,若对连续特征的划分采用了离散化的方法则等同于离散特征}。

分类树中常用的特征选择方法:
\begin{itemize}
\item 信息增益。通过比较划分前与划分后数据集的熵来选择特征。$Gain(D, A_g) = H(D) - \sum_{i=1}^{|A_g|} \frac{|D_i|}{|D|} H(D_i)$。其中$H(\cdot)$表示信息熵,$D_i$表示以$A_g$的取值对$D$进行划分。很显然,第一项为划分前数据集的熵\footnote{$0 \leq H(D) \leq \log_2 |C|$,可以通过朗格朗日来求带约束的最大最小值证明,熵越小表明纯度越高},$H(D) = -\sum_{c=1}^{|C|} \frac{|D_c|}{|D|} \log \frac{|D_c|}{|D|}$,其中$D_c$表示类别为$c$的样本。当然,对于$H(D_i)$或$H(D_c)$,也是使用$C$来计算熵。信息增益对特征的取值数量敏感,偏向于选择取值多的特征。选择使$Gain(D, A_g)$最大的$A_g$来划分,$ID_3$以信息增益来做特征选择的。
\item 信息增益率。解决信息增益偏向取值多的特征的问题,$Gain\_ratio(D, A_g) = \frac{Gain(D, A_g)}{H_{A_g}(D)}$。为了矫正信息增益的问题,信息增益率将信息增益除以了特征$A_g$的固有值(intrinsic value)。$H_{A_g}(D) = -\sum_{i=1}^{|A_g|} \frac{|D_i|}{|D|} \log \frac{|D_i|}{|D|}$。显然,$H_{A_g}(D)$是对$Gain(D, A_g)$的类似于归一化的操作,取值数越多的特征,其$H_{A_g}(D)$越大。选择使$Gain\_ratio(D, A_g)$最大的$A_g$来划分,$C4.5$使用增益率做特则选择。
\item 基尼指数。$D$的基尼指数为$Gini(D) = \sum_{c}^{C} \sum_{c' \neq c}^{C} p_c p_{c'} = \sum_{c}^{C} p_c (1 - p_c) = 1 - \sum_{c}^{C} p_c^2$,同样,基尼指数越小,纯度越高/不确定性越低。$D$在$A_g$下的基尼指数$Gini(D, A_g) = \sum_{c=1}^{|C|} \frac{|D_c|}{|D|} Gini(D_c)$,选择\textbf{使$Gini(D, A_g)$最小的特征$A_g$来划分}。$CART$决策树使用基尼指数做特征选择。
\end{itemize}

\tbc{red}{注意:以上特征选择方法对连续特征和离散特征都适用,并不影响$H(\cdot)$的计算,关键在于是针对分类任务还是回归任务。}

通常,最终的决策树是经过剪枝生成的。生成过程中的剪枝(预估结点划分前后带来的收益,判断是否继续划分)称为\textbf{预剪枝},生成完整的决策树后再剪枝称为\textbf{后剪枝}(自底向上对非叶子节点进行评估,是否将以该结点为根节点的子树替换为叶子节点)。
\begin{itemize}
\item \textbf{预剪枝}。可以通过设置收益阈值,或者使用验证集评估划分前后的预测效果来决定是否对结点进行划分(即是否以该节点为根节点继续划分)。或者,该节点内样本是否都属同一类,节点内样本数量是否小于阈值,类别分布独立于可用特征(即划分到该节点时的$A$),树是否达到了一定高度。
\item \textbf{后剪枝}。递归地从最后一个内部节点开始评估,评估该节点剪枝前后决策树在验证集上的表现,如果没有提高则不对该内部节点剪枝,否则将其收缩为叶子节点,在新的决策树上继续简直。
\end{itemize}

\subparagraph{回归决策树}显然,回归决策树是用来解决回归任务的,预测连续的数值。回归决策树的生成过程与分类决策树很相似,但是有一个比较大的不同:\textbf{回归树通常是二叉树}。回归树也是对特征空间进行划分,每次划分时,选择一个最优的特征及该特征的最优切分点,根据切分点将样本集划分为两个子集,继续分别在两个子集上进行划分(\textbf{在两个子集上依然会使用上一层使用过的特征,但是由于子集中该特征的值都小于某个阈值,所以虽然可能会多次使用一个特征,但是每次的切分点不一样}),最终每个子空间上有一个回归值。

\subparagraph{常见问题}
\begin{myenumerate}
\item 生成决策树过程中,如何处理某个特征值缺失的样本?可以将该特征上缺失值的样本同时划入到所有子结点中,或者专门增加一个子节点,对应特征值缺失的样本。
\item 
\end{myenumerate}


\subsection{t-SNE}
t-distributed Stochastic Neighbor Embedding。一种数据降维的方法。

参考资料:
\begin{enumerate}
	\item \href{https://scikit-learn.org/stable/modules/manifold.html#t-distributed-stochastic-neighbor-embedding-t-sne}{sklearn 中关于tSNE的介绍}
	\item \href{https://www.jianshu.com/p/700f017cd330}{tSNE降维原理}
\end{enumerate}

\subsection{Spectral cluster}
谱聚类。将每个样本视作某个空间中的点。

参考资料:
\begin{enumerate}
	\item \href{https://www.cnblogs.com/pinard/p/6221564.htm}{谱聚类原理总结}
\end{enumerate}


\subsection{变分贝叶斯}
用来近似计算复杂积分,在这类模型中一般包含三类变量:观测变量、未知参数、隐变量,其中位置参数和隐变量统称为不可观测变量。变分贝叶斯的目的主要有两个:
\begin{itemize}
	\item 近似估计不可观测变量的后验概率,以便通过这些变量做出推断
	\item 对于一个特定的模型,给出观测变量边缘似然函数的下界,作为模型选择的依据。一般认为似然概率越高,模型效果越好
\end{itemize}
通常情况下,我们会有一组观测数据(D),那么怎么获得不可观测变量(Z))的后验概率P(Z | D)呢?

通常不可观测变量的后验概率是很复杂的,难以直接计算之,但我们可以先假设一个分布Q(Z)与P(Z|D)是近似的。对于衡量两个分布的差异,可以使用KL散度,即:KL(Q(Z) | P(Z|D)) 。

$$
\begin{equation}\nonumber
	\begin{aligned}
		KL(Q(Z) || P(Z|D)) &= \sum_{z \in Z} Q(z) \log \frac{Q(z)}{P(z|D)} \\
		&= \sum_{z \in Z} Q(z) \log \frac{Q(z) P(D) )}{P(z, D) } \\	
		&= \sum_{z \in Z} Q(z) ( \log \frac{Q(z)}{P(z|D)} + \log P(D) ) \\
		&= \log P(D) + ( \sum_{z \in Z} Q(z) \log \frac{Q(z)}{P(z, D)}) ) 
	\end{aligned}
\end{equation}
$$

显然只要最小化KL散度即可,因为$\log P(D)$是一个常数,所以只要最小化$\sum_{z \in Z} Q(z) \log \frac{Q(z)}{P(z, D)})$即可。将$\sum_{z \in Z} Q(z) \log \frac{Q(z)}{P(z, D)})$记为$\mathcal{L}$,则$\log P(D) = -\mathcal{L} + KL(Q || P)$。显然KL散度一定是非零的,所以$\log P(D)$的下界就是$-\mathcal{L}$。


\subsection{自编码器/变分自编码器}
自动编码器是一种无监督的神经网络,用于学习输入数据的低维表示,并能够根据低维表示重建输入数据,也是一种将为的手段。

变分自动编码(Variational Auto Encoder)器运用了变分贝叶斯的思想,也继承了自动编码器的结构,但是存在一定的差别。在VAE中,默认将输入数据的低维表示(即隐变量)服从某个分布,一般认为服从正态分布。在encoder阶段学习到隐表示的分布 --- 通常是学习到假定分布的参数,得到分布后就从该分布中进行采样得到隐表示;decoder时,则将隐表示输入到decoder中。

参考资料:
\begin{itemize}
	\item \href{https://www.cnblogs.com/kexinxin/p/9858525.html}{变分自动编码器}
	\item \href{https://github.com/cdoersch/vae_tutorial}{vae\_tutorial}
\end{itemize}

\subsection{SVM}
一种二分类模型,定义在特征空间上的间隔最大化的线性分类器,通过间隔最大化找到支持向量来完成算法的学习。

给定数据集:$D = \{(x_1, y_1), (x_2, y_2), ..., (x_N, y_N)\}, x_i \in \mathcal{X} = \mathbb{R}^n, y_i \in \mathcal{Y} = \{-1, +1\}, i = 1, 2, ..., N$。目的是找到最优的超平面$w^T x + b = 0$对$D$进行划分。

\paragraph{间隔最大化}如果一个\textbf{超平面能够正确划分所有样本},则有:
$$
\begin{cases}
	{w}^{\mathrm{T}} {x}_{i}+b > 0, & y_{i}=+1 \\ 
	{w}^{\mathrm{T}} {x}_{i}+b < 0, & y_{i}=-1
\end{cases}
$$ 
从上式可以得出,分别存在$0 < \alpha = min(\{w^T x_i + b | y_i = +1\}), 0 > \beta = max(\{w^T x_i + b | y_i = -1\})$使,
$$
\begin{cases}
	{w}^{\mathrm{T}} {x}_{i}+b \geqslant \alpha, & y_{i}=+1 \\ 
	{w}^{\mathrm{T}} {x}_{i}+b \leqslant \beta, & y_{i}=-1
\end{cases}
$$
对$(w, b)$进行缩放,令$w = \frac{w}{min(|\alpha|, |\beta|)}, b = \frac{b}{min(|\alpha|, |\beta|)} $,则有:
$$
\begin{cases}
	{w}^{\mathrm{T}} {x}_{i}+b \geqslant +1, & y_{i}=+1 \\ 
	{w}^{\mathrm{T}} {x}_{i}+b \leqslant -1, & y_{i}=-1
\end{cases}
$$
其中,使上式等号成立的样本点为\textbf{支持向量}。

由于划分超平面有很多,直觉上,我们希望划分超平面能够尽可能位于两类样本中间,即间隔两类样本最大超平面,SVM中是这样定义间隔的:\textbf{两个异类支持向量到超平面的距离之和},即:
$$
\gamma = \frac{2}{||w||}
$$
$\gamma$就是\textbf{\textcolor{red}{间隔}},间隔最大化的物理意义:
\begin{itemize}
	\item 间隔越大,表示分类的置信度越大,对hard样本也能够有较高的分类置信度
	\item 间隔越大,泛化性能越好
	\item 间隔越大,对噪声的鲁棒性更强
\end{itemize}


\paragraph{间隔最大化的最优化目标}
\begin{align}
	\mathop{max}_{w, b}&\quad \frac{2}{||w||} \nonumber \\
	s.t.&\quad y_i(w^T x_i + b) \geqslant 1, i = 1, 2, ..., N \nonumber
\end{align}
对上式进行简单的转化,可得:
\begin{align}
	\mathop{min}_{w, b}&\quad \frac{1}{2}||w||^2 \nonumber \\
	s.t.&\quad y_i(w^T x_i + b) \geqslant 1, i = 1, 2, ..., N \nonumber
\end{align}
优化目标中的约束,就是我们的期望:超平面能够将所有样本正确分类,这也是我们的一个假设,\textcolor{red}{\textbf{数据集$D$是线性可分的!}}。但很不幸,很多情况$D$都不是线性可分的!

\textbf{\checkmark 2021-11-18}\\
\paragraph{SVM优化目标推导的另一形式}
给定分离超平面$w^T x + b = 0$,$\forall x_i \in D$,$x_i$与分离平面的距离$d(w, b; x_i) = |w^T x_i + b|$,则$D$与分离超平面的距离为其中的最小者,及$d(w, b; D) = \mathop{min}_{x_i} |w^T x_i +b|$,因为$w, b$按比例变化后分离超平面并不变但会带来距离的变化,故令$d(w, b; D) = \mathop{min}_{x_i} \frac{|w^T x_i + b|}{||w||}$。若分离超平面能对所有样本正确分类,则会有$y_i ( w^T x_i + b ) \geq 0$,且显然$y_i ( w^T x_i + b ) \geq \mathop{min}_{x_i} |w^T x_i + b|$,由于可以改变$w, b$来控制距离,则令$\mathop{min}_{x_i} |w^T x_i + b| = 1$,则得到了$y_i ( w^T x_i + b ) \geq 1,\ d(w, b; D) = \mathop{min}_{x_i} \frac{1}{||w||}$。但是为了让$d(w, b; D)$尽量大(以保证置信度和鲁棒性),因此,优化目标为:
\begin{align}
	\mathop{max}_{w, b}&\quad \frac{2}{||w||} \nonumber \\
	s.t.&\quad y_i(w^T x_i + b) \geqslant 1, i = 1, 2, ..., N \nonumber
\end{align}

\textbf{\checkmark 2021-11-18}\\
\paragraph{SVM优化目标推导的再一形式}
由上述可知,显然有$y_i ( w^T x_i + b ) \geq 1$,则分离超平面与两类样本的距离为:
\begin{align}
	d(w, b) &= \mathop{min}_{x_i, y_i=-1} d(w, b; x_i) + \mathop{min}_{x_i, y_i=+1} d(w, b; x_i) \nonumber \\
			&= 	\mathop{min}_{x_i, y_i=-1} \frac{|w^T x_i + b|}{||w||} + \mathop{min}_{x_i, y_i=+1} \frac{|w^T x_i + b|}{||w||} \nonumber \\
			&= \frac{1}{||w||} ( \mathop{min}_{x_i, y_i=-1} |w^T x_i + b| + \mathop{min}_{x_i, y_i=+1} |w^T x_i + b|) \nonumber \\
			&= \frac{2}{||w||}
\end{align}
后续就不必多言了。



\paragraph{线性可分SVM的求解}
可以应用拉格朗日对偶性来求解,将线性可分得SVM最优化问题看作原始问题,通过求解对偶问题来求解原始问题。则原始问题为:
\begin{align}
	\mathop{min}_{w, b}&\quad \frac{1}{2}||w||^2 \nonumber \\
	s.t.&\quad 1 - y_i(w^T x_i + b) \leq 0, i = 1, 2, ..., N \nonumber
\end{align}
转换拉格朗日函数:
$$
L(w, b, \alpha) = \frac{1}{ 2}  ||w||^2 + \sum_{i=1}^{N} \alpha_i (1  - y_i(w^T x_i + b))
$$
根据拉格朗日对偶性,原始问题的对偶问题是极大极小问题:
$$
\mathop{max}_{\alpha} \mathop{min}_{w, b} L(w, b, \alpha)
$$
那么求解过程可以分为:
\begin{myenumerate}
	\item 固定$\alpha$,求$\mathop{min}_{w, b} L(w, b, \alpha)$。这个通过求偏导并令其等于0,可得:
	\begin{align}
		w &= \sum_{i=1}^N \alpha_i y_i x_i	\nonumber \\
		&\sum_{i=1}^N \alpha_i y_i = 0 \nonumber
	\end{align}
	
	\item 将$w, b$用$\alpha$相关的式子表示,求$\mathop{min}_{w, b} L(w, b, \alpha)$对$\alpha$的极大。代入$L$后得到的:
	$$
	L(w, b, \alpha) = -\frac{1}{2} \sum_{i=1}^N \sum_{j=1}^{N} \alpha_i \alpha_j y_i y_j (x_i \cdot x_j) + \sum_{i=1}^N \alpha_i
	$$	
	则对偶问题成了(转换成了$min$):
	\begin{align}
		\mathop{min}_{\alpha}\quad &\frac{1}{2} \sum_{i=1}^N \sum_{j=1}^{N} \alpha_i \alpha_j y_i y_j (x_i \cdot x_j) - \sum_{i=1}^N \alpha_i \nonumber \\
		s.t.\quad &\sum_{i=1}^N \alpha_i y_i = 0 \nonumber \\
				  &0 \leq \alpha_i \leq C, i = 1, 2, ..., N \nonumber
	\end{align}
	原始问题满足相关的条件,故原始问题的最优解$w^*, b^*$可以通过对偶问题的最优解$\alpha^*$来得到。先假设已经解得了$\alpha^*$,则显然有:
	$$
	w^* = \sum_{i=1}^N \alpha_i^* y_i x_i
	$$
	那么$b$呢?回忆KKT条件\ref{kkt},有$\alpha_i^* (1 - y_i(w^* x_i + b^*)) = 0$,因为$\alpha \neq 0$,故肯定存在$\alpha_j^* \neq 0$,则:
	\begin{align}
		&1 - y_j(w^* x_j + b^*) = 0 \nonumber \\
		&\mathop{\Longrightarrow}_{\times y_j} y_j - y_j^2 (w^* x_j + b^*) = 0 \nonumber \\
		&\mathop{\Longrightarrow}_{rep.\  w^*} y_j - 1 \cdot (\sum_{i=1}^N \alpha_i^* y_i x_i x_j + b^*) = 0 \nonumber \\
		&\Longrightarrow b^* = y_j - \sum_{i=1}^N \alpha_i^* y_i (x_i \cdot x_j) \nonumber 
	\end{align}
	注意$y_j$是$\alpha_j^* \neq 0$所对应的样本的标签,显然,\textcolor{red}{$(x_j, y_j)$就是那些支持向量},同理,\textcolor{red}{在计算$w^*$时起作用的是全部的支持向量}。
	现在,任务就是如何求出$\alpha^*$了!	求解$\alpha^*$是另一个优化问题了,通常使用\textit{Sequential Minimal Optimization, SMO}\ref{smo}求解。
\end{myenumerate}

至此,得到了$w^*, b^*$,分离超平面也就确定了$w^* \cdot x + b^* = 0$,那么也就可以开始使用SVM了,分类决策函数:
$$
f(x) = sign(w^* \cdot x + b^*)
$$

在上述过程中,是否还记得一个假设:\textbf{\textcolor{red}{数据集$D$是线性可分的!}}这个假设对我们的求解过程有什么影响呢?\textbf{\textcolor{red}{不等式约束}}。对于线性可分的SVM,是在\textbf{硬间隔最大化}的基础上构建的。显然,现实世界没有那么简单!

\paragraph{线性SVM}
这个时候的数据集$D$,并不是完全线性可分的,但是除去数据中的一些异常点后,剩下的点是线性可分的。这个时候考虑如何将异常点带来的影响融入到模型中。

若允许一部分异常点被分错,即
$$
\begin{cases}
	{w}^{\mathrm{T}} {x}_{i}+b > \alpha_i \leq 0, & y_{i}=+1 \\ 
	{w}^{\mathrm{T}} {x}_{i}+b < \beta_i \geq 0, & y_{i}=-1
\end{cases}
$$ 
这个式子的含义是:对于异常点,允许其在超平面的另一边。经过以下转化:
\begin{quotation}
	$$
	\begin{cases}
		{w}^{\mathrm{T}} {x}_{i}+b - \alpha_i > 0, & y_{i}=+1 \\ 
		{w}^{\mathrm{T}} {x}_{i}+b - \beta_i < 0, & y_{i}=-1
	\end{cases}
	$$
	存在$u >0, v < 0$,使得:
	$$
	\begin{cases}
		{w}^{\mathrm{T}} {x}_{i}+b - \alpha_i \geq u, & y_{i}=+1 \\ 
		{w}^{\mathrm{T}} {x}_{i}+b - \beta_i \leq v, & y_{i}=-1
	\end{cases}
	$$
	经过缩放后(注意$\alpha_i, \beta_i$也缩放了),有:
	$$
	\begin{cases}
		{w}^{\mathrm{T}} {x}_{i}+b - \alpha_i \geq 1, & y_{i}=+1 \\ 
		{w}^{\mathrm{T}} {x}_{i}+b - \beta_i \leq -1, & y_{i}=-1
	\end{cases}
	$$
	移项,有
	$$
	\begin{cases}
		{w}^{\mathrm{T}} {x}_{i}+b \geq 1 + \alpha_i, & y_{i}=+1 \\ 
		{w}^{\mathrm{T}} {x}_{i}+b \leq -1 + \beta_i , & y_{i}=-1
	\end{cases}
	$$
	左右两边分别乘对应的$y_i$,同时令\textcolor{red}{$\alpha_i = -\alpha_i$},此时,\textbf{\textcolor{red}{$\alpha_i \geq 0, \beta_i \geq 0$}},则有:
	$$
	\begin{cases}
		y_i( {w}^{\mathrm{T}} {x}_{i}+b ) \geq 1 - \alpha_i, & y_{i}=+1 \\ 
		y_i( {w}^{\mathrm{T}} {x}_{i}+b) \textcolor{red}{\geq} 1 - \beta_i , & y_{i}=-1
	\end{cases}
	$$	
\end{quotation}
针对每个样本$(x_i, y_i)$,给$\alpha_i, \beta_i$取个新的名字$\xi_i$(\textbf{松弛变量}),则有:
$$
y_i( {w}^{\mathrm{T}} {x}_{i}+b ) \geq 1 - \xi_i
$$
是否感觉\textbf{\textcolor{red}{又回到了最初的起点}}?

注意,松弛变量是针对每个样本的,即每个样本都有一个自己的松弛变量,$1-\xi_i$的物理含义是$x_i$里分离超平面的距离,$\xi_i$就是$x_i$离由距分离超平面为1的同类样本点组成的平面的距离,所以,\textbf{理所当然的希望$\xi_i$尽量小},这也是希望异常点不要太异常。

因此,很容易可以得到新的优化目标,其中$C > 0$是\textbf{惩罚参数}:
\begin{align}
	\mathop{min}_{w, b, \xi}&\quad \frac{1}{2} ||w||^2 + C \sum_{i=1}^{N} \xi_i \nonumber \\
	s.t.&\quad y_i(w^T x_i + b) \geqslant 1 - \xi_i, i = 1, 2, ..., N \nonumber \\
		&\quad \xi_i \geq 0, i = 1, 2, ..., N \nonumber
\end{align}
该问题的对偶问题为:
\begin{align}
	\mathop{min}_{\alpha}\quad &\frac{1}{2} \sum_{i=1}^N \sum_{j=1}^{N} \alpha_i \alpha_j y_i y_j (x_i \cdot x_j) - \sum_{i=1}^N \alpha_i \nonumber \\
	s.t.\quad &\sum_{i=1}^N \alpha_i y_i = 0 \nonumber \\
	&0 \leq \alpha_i \leq C, i = 1, 2, ..., N \nonumber
\end{align}
求解这个约束优化问题,同样利用拉格朗日对偶性,求解过程与线性可分SVM类似,不再赘述,且其解为:
$$
w^* = \sum_{i=1}^N \alpha_i^* y_i x_i
$$
$$
b = y_j - \sum_{i=1}^N \alpha_i y_i (x_i \cdot x_j)
$$
但有一些不同:\textcolor{red}{\textbf{支持向量}}。经过类比,很显然,线性SVM中的支持向量应满足:$1 - \xi_j = y_j ( w^* x_j + b^* ) $,即支持向量与分离超平面的距离不一定是1,而是$1 - \xi_j$。

\subparagraph{合页损失函数}
SVM希望所有样本尽量满足$y_i( w^T x_i + b ) \geq 1$,考虑损失函数:
$$
\mathop{min}_{w, b} \frac{1}{2} ||w||^2 + C \sum_{i=1}^{N} max(0, 1 - y_i (w^T x + b))
$$
定义\textit{hinge-loss}(合页损失)$l_{hinge}(z) = max(0, 1-z)$。$l_{hinge}$不仅要求分类正确,还要求有较高的置信度(离分类超平面有一定距离)时损失才为0。$y_i( w^T x_i + b )$表示$x_i$离超平面的距离,\textbf{当错误分类时,显然损失大于0;当正确分类时,显然有$y_i( w^T x_i + b ) \geq 0$,但这并不能保证这个样本带来的损失为0,只有当其处于超平面正确的一侧,且距离大于1($y_i( w^T x_i + b ) \geq 1$)时损失才为0}。

令$max(0, 1-y_i (w^T x_i + b)) = \xi_i$,显然$\xi_i \geq 0$,且当$1-y_i (w^T x_i + b) > 0$时,$1-y_i (w^T x_i + b) = \xi_i$,当$1-y_i (w^T x_i + b) \leq 0$时,$\xi_i = 0$,即:
$$
1-y_i (w^T x_i + b) \leq \xi_i \Longrightarrow y_i (w^T x_i + b) \geq 1 - \xi_i
$$
所以上述损失函数转化为:
\begin{align}
	\mathop{min}_{w, b, \xi}&\quad \frac{1}{2} ||w||^2 + C \sum_{i=1}^{N} \xi_i \nonumber \\
	s.t.&\quad y_i(w^T x_i + b) \geqslant 1 - \xi_i, i = 1, 2, ..., N \nonumber \\
	&\quad \xi_i \geq 0, i = 1, 2, ..., N \nonumber
\end{align}
可见,合页损失等价于一个约束优化问题。

\paragraph{非线性SVM}
在输入空间中,不能找到一个超平面对不同的类别进行分类。如果要用SVM来分类的话,线性模型的SVM该怎么办呢 --- \textbf{将样本空间映射到一个另一个空间中,目标空间中的样本点是线性可分的}。那么利用SVM对于线性不可分的$D$进行分类的问题就成了:1)空间映射;2)求解SVM相关的优化问题。

\subparagraph{核技巧}
核技巧就是用来进行空间映射的方法。数据集$D$的源空间为$\mathcal{X}$,目标空间为$\mathcal{H}$。定义$\mathcal{X}$到$\mathcal{H}$的映射:
$$
\phi (x): \mathcal{X} \rightarrow \mathcal{H}
$$
经过$\phi(\cdot)$的映射后的$D$将变成线性的(注意不一定是线性可分的),则根据线性SVM中的对偶问题可得其目标函数为(省略了约束条件):
$$
\mathop{min}_{\alpha}\quad &\frac{1}{2} \sum_{i=1}^N \sum_{j=1}^{N} \alpha_i \alpha_j y_i y_j (\phi(x_i) \cdot \phi(x_j)) - \sum_{i=1}^N \alpha_i
$$
那么,如何找到映射函数呢?通常,并不直接定义映射函数,而是间接定义一个核函数,用来表示两个样本点在目标空间中的内积,
$$
\kappa(x_i, x_j) = \phi(x_i) \cdot \phi(x_j)
$$
因此,在求解时涉及到的样本点的内积都可以用和函数来代替了。
\textbf{\textcolor{red}{怎么找核函数呢?}}\ref{pdkf}

\subparagraph{求解非线性SVM}
参考线性SVM中原始问题的对偶问题,可以写出非线性SVM的对偶问题:
\begin{align}
	\mathop{min}_{\alpha}\quad &\frac{1}{2} \sum_{i=1}^N \sum_{j=1}^{N} \alpha_i \alpha_j y_i y_j \kappa(x_i, x_j) - \sum_{i=1}^N \alpha_i \nonumber \\
	s.t.\quad &\sum_{i=1}^N \alpha_i y_i = 0 \nonumber \\
	&0 \leq \alpha_i \leq C, i = 1, 2, ..., N \nonumber
\end{align}
此时,可得其解为:
$$
w^* = \sum_{i=1}^N \alpha_i^* y_i x_i
$$
$$
b = y_j - \sum_{i=1}^N \alpha_i y_i \kappa(x_i, x_j)
$$
则决策函数可写成:
$$
f(x) = sign(\sum_{i=1}^N \alpha_i^* y_i \kappa(x_i, x)) + b^*
$$

\paragraph{支持向量回归}
Support Vector Regression,一种线性的回归模型。此时,数据集$D = \{(x_1, y_1), (x_2, y_2), ..., (x_N, y_N)\}, x_i \in \mathbb{R}^n, y_i \in \mathbb{R}$。要学习的模型为:$f(x) = w^T x + b$。通过损失函数的角度来引出SVR。

通常的回归模型采用的是均方误差函数,只有在预测值与真实值完全一样时损失才为0。SVR假设能够容忍模型有$\epsilon$的误差,可以定义损失函数:
$$
l_\epsilon (z) = \begin{cases}0, &|z| \leq z \\ |z| - \epsilon, &others \end{cases}
$$
则SVR转化为:
$$
\mathop{min}_{w, b} \frac{1}{2} ||w||^2 + C \sum_{i=1}^N l_\epsilon (f(x_i) - y_i) 
$$
类比于合页损失,在$f(x)$两侧引入不同的松弛变量$\xi_i, \hat{\xi_i}$,则可将上述目标转化为优化问题:
\begin{align}
	\mathop{min}_{w, b, \xi, \hat{\xi}}\quad &\frac{1}{2} ||w||^2 + C \sum_{i=1}^{N} (\xi_i + \hat{\xi_i}) \nonumber \\
	s.t.\quad &f(x_i) - y_i \leq \epsilon + \xi_i \nonumber \\
			  &y_i - f(x_i) \leq \epsilon + \hat{\xi_i} \nonumber \\
			  &\xi_i \geq 0, \hat{\xi_i} \geq 0, i = 1, 2, ..., N \nonumber
\end{align}
后续求解与上述类似。


\subsection{PCA}
全称 Principal component analysis,即主成分分析。PCA 利用正交变换把由线性相关变量表示的观测数据转换为少数几个线性无关变量表示的数据,少数的线性无关变量就是主成分。如果从基的角度来看,就是将数据在原来的相关的基下的表示,转化为无关的基下的表示。

一个简单的流程:
\begin{enumerate}
	\item 给定数据矩阵 $X \in \mathbb{R}^{m \times n}$,其中 $m$ 表示特征数,$n$ 表示样本数;
	
	\item 对 $X$ 去中心化后得到 $\hat{X}$;
	
	\item 求 $\hat{X} \hat{X}^T$ (相当于 $X$ 的协方差矩阵)的特征向量,取特征指最大的前 $k$ 个特征向量组成矩阵 $P \in \mathbb{R}^{k \times m}$。$P$ 就是用于降维的矩阵;
	
	\item 对 $\hat{X}$ 进行降维:$P \hat{X}$;
\end{enumerate}


\subsection{metric learning} 
度量学习。学习如何衡量两个对象之间的相似度/距离。

\subsection{GBDT 与 XGBoost 的对比}
首先,两种都是 boosting 方法,加法模型,都是迭代地去拟合目标值与现有预测值的残差。它们之间的区别:
\begin{itemize}
	\item xgboost 对损失函数进行了二阶展开,利用了一阶、二阶的信息,以牛顿法求解参数;GBDT 只用到了一阶导信息;
	
	\item xgboost 支持自定义代价函数,只需要提高损失函数的一、二阶导即可;
	
	\item GBDT 以 CART 作为基分类器,xgboost 除了以树为基分类器,还可以以线性模型作为基分类器;
	
	\item xgboost 中加入了正则化项;GBDT中没有;
	
	\item xgboost 中引入了列抽样和行采样;
	
	\item xgboost 能够自动学习处缺失值的分裂方向;
	
	\item xgboost 支持节点级别的并行,并且支持直方图算法;
\end{itemize}

\subsection{SMOTE}
SMOTE(Synthetic Minority Over-Sampling Technique) 是一种综合采样人工合成数据算法,用于解决数据类别不平衡问题,以 Over-sampling 少数类和 Under-sampling 多数类结合的方式来合成数据。

SMOTE 以每个样本点的 k 个最近邻样本点为依据,随机的选择 N 个邻近点,样本点与邻近点进行差值再乘上一个 [0,1] 范围的阈值,从而达到合成数据的目的。这种算法的核心是:特征空间上邻近的点其特征都是相似的。它并不是在数据空间上进行采样,而是在特征空间中进行采样,所以它的准确率会高于传统的采样方式。这也是为什么到目前为止 SMOTE 以及其派生的算法仍然是较为主流的采样技术的原因。