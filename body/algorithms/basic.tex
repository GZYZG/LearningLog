\subsection{常见排序算法复杂度}
% Please add the following required packages to your document preamble:
% \usepackage[table,xcdraw]{xcolor}
% If you use beamer only pass "xcolor=table" option, i.e. \documentclass[xcolor=table]{beamer}
\begin{table}[http]
	\begin{tabular}{cccccc}
		\cline{1-4}
		\rowcolor[HTML]{4371C3} 
		\multicolumn{1}{|c|}{\cellcolor[HTML]{4371C3}{\color[HTML]{FFFFFF} \textbf{算法}}} & \multicolumn{1}{c|}{\cellcolor[HTML]{4371C3}{\color[HTML]{FFFFFF} \textbf{时间复杂度(平均)}}} & \multicolumn{1}{c|}{\cellcolor[HTML]{4371C3}{\color[HTML]{FFFFFF} \textbf{时间复杂度(最坏)}}} & \multicolumn{1}{c|}{\cellcolor[HTML]{4371C3}{\color[HTML]{FFFFFF} \textbf{时间复杂度(最好)}}} & {\color[HTML]{FFFFFF} 空间复杂度} & {\color[HTML]{FFFFFF} 稳定性} \\ \cline{1-4}
		
		\textbf{直接插入排序} & $O(n^2)$ & $O(n^2)$ & $O(n)$ & $O(1)$ & 稳定 \\ \hline
		\textbf{希尔排序} & $O(n^{1.3})$ & $O(n^2)$ & $O(n)$ & $O(1)$ & 不稳定 \\ \hline
		\textbf{直接选择} & $O(n^2)$ & $O(n^2)$ & $O(n^2)$ & $O(1)$ & 不稳定 \\ \hline
		\textbf{堆排序} & $O(n \log_2 n)$ & $O(n \log_2 n)$ & $O(n \log_2 n)$ & $O(1)$ & 不稳定 \\ \hline
		\textbf{冒泡排序} & $O(n^2)$ & $O(n^2)$ & $O(n)$ & $O(1)$ & 稳定 \\ \hline
		\textbf{快速排序} & $O(n \log_2 n)$ & $O(n^2)$ & $O(n \log_2 n)$ & $O(\log_2 n)$ & 不稳定 \\ \hline
		\textbf{归并排序} & $O(n \log_2 n)$ & $O(n \log_2 n)$ & $O(n \log_2 n)$ & $O(n)$ & 稳定 \\ \hline
		\textbf{基数排序} & $O(d(n+r))$ & $O(d(n+r))$ & $O(d(n+r))$ & $O(r)$ & 稳定 \\ \hline
		\textbf{计数排序} & $O(n+k)$ & $O(n+k)$ & $O(n+k)$ & $O(n+k)$ & 稳定 \\ \hline
	\end{tabular}
\end{table}

注意, 基数排序中的 $d, r$ 分别表示位数和基数;\href{https://blog.csdn.net/alzzw/article/details/98245871}{计数排序}中 $k$ 是整数的范围. 

\subsection{大整数}
\paragraph{Motivation}在C、C++等一些语言中, 支持的整数类型所能保存的范围是有效的, 相对Java、Python等一些语言中支持的范围是要小的. 

\paragraph{HOW?}考虑几个可能的涉及大整数的例子: 大整数的加法、大整数的减法大整数乘某个数大整数除以某个数. 主要涉及到的问题有: 
\begin{itemize}
	\item 大整数的存储
	\item 大整数之间的加减法
	\item 大整数与一个常规整数的乘除法
\end{itemize}


\subsection{前缀和与差分}


\subsection{并查集}
\paragraph{Motivation}将n个不同的元素分成一组\textbf{不相交}的集合. 能够快速地完成两种操作: 查找某个元素属于那个集合、合并两个集合. 注意: “不相交”可以有不同的定义, 如graph中的节点是否相邻、在某种度量下的不相交等. 

\paragraph{HOW?}

\subsection{链表}
链表应该算是使用非常广泛的一种数据结构了. 先从最简单的链表说起: 单链表. 链表中的组成元素是一个个的节点, 节点中包含了指向下一个节点的指针 \mintinline{python}{next} 以及所包含的数据. 定义一下链表的节点: 
\begin{python}
	class ListNode:
	     def __init__(self, val=0, next=None):
	         self.val = val
	         self.next = next
\end{python}
对于一个链表, 是可以包含多个节点的, 节点之间通过 \mintinline{python}{next} 指针相连, 即可以从一个节点沿着 \mintinline{python}{next} 遍历到下一个节点. 

\subsection{布隆过滤器}
布隆过滤器 (Bloom Filter) 是一个高空间利用率的概率性数据结构, 由一个二进制向量 (即位数组) 和一系列随机映射函数 (即哈希函数) 两部分组成, 能高效地查询一个元素是否在一个集合中 --- 这也是布隆过滤器的主要作用. 

简单介绍一下它的工作原理. 如果用普通的线性数据结构, 如数组, 来检索一个元素是否存在时, 则需要遍历整个数据结构. 如何避免遍历呢? 借助哈希表, 构建值与存储位置的关系. 布隆过滤器就借助了哈希. 布隆过滤器包含一个数组和 $k$ 个哈希函数. 数组初始状态是全 0 的, 每个哈希函数为一个元素计算出一个索引值. 在插入 $x$ 时, $k$ 个哈希函数计算出 $k$ 个索引值, 将数组这 $k$ 个位置处的值置 1. 检索 $x$ 是否存在时, 先计算 $k$ 个索引, 若这些位置都为 1 则认为 $x$ 存在, 否则认为不存在.

注意在检索时, 若 $k$ 个位置不都为 1 则肯定可以认为元素不存在. 但是, 都为 1时却不能保证 $x$ 一定存在, 即存在一定的误判率. 这个误判率和数组的长度 ($m$), 已添加的元素的数量 ($n$), 哈希函数的数量 ($k$) 的关系是什么呢? 

假设选择的哈希函数的输出是等概率的. 那么插入一个元素时, 某个位置被一个哈希函数选中 (取 1) 的概率为 $\frac{1}{m}$, 不被选中 (取 0) 的概率为 $1 - \frac{1}{m}$, 则 $k$ 个哈希函数都没选中该位置的概率为 $(1 - \frac{1}{m})^k$. 如果插入了 $n$ 个元素, 该位置仍然为 0 的概率为 $(1 - \frac{1}{m})^{kn}$, 该位置为 1 的概率为 $1 - (1 - \frac{1}{m})^{kn}$. 误判的情况是 $x$ 对应的 $k$ 个位置已经被之前插入的 $n$ 个元素置 1 了, 这个概率是多大呢? \textbf{$(1 - (1 - \frac{1}{m})^{kn})^k \approx (1 - e^{- \frac{k n}{m}})^k$}.

对于给定的 $m, n$, 使误判率最小时 $k = \frac{m}{n} \ln 2$. 令误判率为 $P_{fp}$, 对于给定的 $P_{fp}, n$, 最优的 $m = - \frac{n \ln P_{fp}}{(\ln 2)^2}$. 

布隆过滤器的特点: 1) 相比哈希表, 空间和时间效率都很高; 2) \textcolor{red}{\textbf{能够准确地判断元素不存在}} (这一点可以加以应用, 哪些场景是需要判断不存在的情况, 这样就可以大大减少相应的开销, 即使会有一小撮误判存在的情况), 判断结果为存在时有一定的误报率; 3) 只能插入元素, 不能删除元素.

应用场景: 1) 实现数据的去重, 但可能有一定的误报情况; 2) 数据库防止穿库. 可以把数据表的主键存进布隆过滤器中, 当查询数据时先检测是否存在该主键, 可以准确判断不存在的情况, 这样就能减少磁盘的 IO 操作. 该方法还可以解决缓存穿透的问题, 避免查询缓存中不存在的数据. 