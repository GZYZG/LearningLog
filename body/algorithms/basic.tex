\subsection{大整数}
\paragraph{Motivation}在C、C++等一些语言中,支持的整数类型所能保存的范围是有效的,相对Java、Python等一些语言中支持的范围是要小的。

\paragraph{HOW?}考虑几个可能的涉及大整数的例子:大整数的加法、大整数的减法大整数乘某个数大整数除以某个数。主要涉及到的问题有:
\begin{itemize}
	\item 大整数的存储
	\item 大整数之间的加减法
	\item 大整数与一个常规整数的乘除法
\end{itemize}


\subsection{前缀和与差分}


\subsection{并查集}
\paragraph{Motivation}将n个不同的元素分成一组\textbf{不相交}的集合。能够快速地完成两种操作:查找某个元素属于那个集合、合并两个集合。注意:“不相交”可以有不同的定义,如graph中的节点是否相邻、在某种度量下的不相交等。

\paragraph{HOW?}

\subsection{链表}
链表应该算是使用非常广泛的一种数据结构了。先从最简单的链表说起:单链表。链表中的组成元素是一个个的节点,节点中包含了指向下一个节点的指针 \mintinline{python}{next} 以及所包含的数据。定义一下链表的节点:
\begin{python}
	class ListNode:
	     def __init__(self, val=0, next=None):
	         self.val = val
	         self.next = next
\end{python}
对于一个链表,是可以包含多个节点的,节点之间通过 \mintinline{python}{next} 指针相连,即可以从一个节点沿着 \mintinline{python}{next} 遍历到下一个节点。