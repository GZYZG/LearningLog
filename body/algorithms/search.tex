\subsection{贪心算法}
贪心搜索是一种搜索解空间的策略,在搜索时,基于当前状态,选择代价最低或估计收益最大的方向作为下一步搜索的方向。使用贪心搜索并不能保证全局最优解,目的是为了找到一个可行解。在使用贪心算法前,先要证明以下两个东西:
\begin{itemize}
	\item 贪心选择性:指所求问题的整体最优解可以通过一系列局部最优的选择(通过最优度量标准来判断)来到达
	\item 最优子结构:一个问题的最优解中包含了子问题(比原问题规模更小的问题)的最优解,即每一步贪心选完后会留下子问题,子问题的最优解和贪心选出来的解可以凑成原问题的最优解
\end{itemize}
和动态规划相比,二者都需要描述清除问题的优化子结构;但是贪心算法重点是证明贪心选择的合理性,DP重点是找到子问题和合理的递推关系式。

典型的贪心问题:Huffman编码、活动选择问题。



\subsection{Beam Search}
也叫集束搜索。
对贪心搜索的一种改进。在进行搜索时,并不是只沿一个方向搜索,会同时保留$n$个最优的方向,分别以这$n$个方向进行扩展,扩展后在每个方向都可以得到$n$个备选项,那么则有$n^2$个备选项,在这$n^2$个被选项里选择最优的$n$个备选项作为下次扩展的方向,以此类推,即在搜索树的每一层都只保留$n$个方向。$n$就是beam width。


