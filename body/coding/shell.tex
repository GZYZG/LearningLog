\subsection{变量}
\paragraph{基本概念}
定义 shell 变量时不需要加 \$ 符号, 如 \mintinline{shell}{var=444}, 切记定义变量是等号之间不能有空格, 因为 shell 中将空格看作参数的分隔符, 则会把变量名当作函数名, 把剩下的部分当作参数传进去!

要使用一个定义过的变量时, 只要在变量名前加美元符号即可: \mintinline{shell}{$var}, 或者 \mintinline{shell}{${var}}. 变量名为的花括号是可选的, 加花括号是为了帮助 shell 解释器识别变量的边界, 例如 \mintinline{shell}{"This is ${var}-laboratory!"}

已经定义过的变量是可以重新定义的, 注意定义的时候是不加美元符号的!

\paragraph{只读变量}
使用 \mintinline{shell}{readonly} 命令可以将变量定义为只读变量: \mintinline{shell}{readonly var}.

\subsection{字符串}
\subsubsection{\#*, \#\#*, \%*, \%\%* 的含义及用法}
通过一个示例来解释, 定义一个变量: \mintinline{shell}{file=/dir1/dir2/dir3/my.file.txt}.

\begin{myitemize}
	\item \mintinline{shell}{${file#*/}} : 删除\textbf{第一个} \mintinline{shell}{/} 及其左边的字符串, 故该命令的值为 \mintinline{shell}{dir1/dir2/dir3/my.file.txt};
	
	\item \mintinline{shell}{${file##*/}} : 删除\textbf{最后一个} \mintinline{shell}{/} 及其左边的字符串, 故该命令的值为 \mintinline{shell}{my.file.txt};
	
	\item \mintinline{shell}{${file%/*}} : 删除\textbf{最后一个}(即\textbf{从右往左的第一个}) \mintinline{shell}{/} 及其右边的字符串, 故该命令的值为 \mintinline{shell}{/dir1/dir2/dir3};
			
	\item \mintinline{shell}{${file%%/*}} : 删除\textbf{第一个}(即\textbf{从右往左的最后一个}) \mintinline{shell}{/} 及其右边的字符串, 故该命令的值为 \mintinline{shell}{}(空值);
	
\end{myitemize}
注意, 上述的 \mintinline{shell}{/} 可以是其他字符. 显然, \mintinline{shell}{#} 是删除左边, \mintinline{shell} 表示从右向左匹配, \mintinline{shell}{#} 表示从左向右匹配, 都是非贪婪匹配, 即匹配到符合条件的最短结果. \mintinline{shell}{##}, \mintinline{shell} 则属于贪婪匹配, 即匹配符合条件的最长结果.
\end{myitemize}

\subsubsection{截取字符串}
定义一个字符串: \mintinline{shell}{var=12345678}.

\begin{myitemize}
	\item \mintinline{shell}{${var:OFFSET}}, 从 OFFSET 开始截取字符串. 如 \mintinline{shell}{${var:2}}, 即从下标 2 开始截取字符串 (shell 中下标也是从 0 开始), 结果为 \mintinline{shell}{345678};
	
	\item \mintinline{shell}{${var:OFFSET:LENGTH}}, 从 OFFSET 开始截取长度为 LENGTH 的子串. 如 \mintinline{shell}{${var:2:2}}, 即从下标 2 开始截取长度为 2 的串, 结果为 \mintinline{shell}{34};
	
	\item \mintinline{shell}{${var:0-OFFSET}}, 从导数第 OFFSET 个字符串开始截取串, 类似于 python 中的负的索引. 如 \mintinline{shell}{${var:0-2}}, 即从倒数第 2 个开始截取字符串, 结果为 \mintinline{shell}{78}。 那能不能不写 0 呢? 当然可以, 把 0 换成空格即可, 如 \mintinline{shell}{${var: -OFFSET}}; 或者这样 \mintinline{shell}{${var:(-OFFSET)}};
	
	\item \mintinline{shell}{${var:0-OFFSET:LENGTH}}, 从导数第 OFFSET 个字符串开始截取长度为 LENGTH 的子串. 如 \mintinline{shell}{${var:0-2:1}}, 即从倒数第 2 个字符开始截取长度为 1 的子串. 结果为 \mintinline{shell}{7};
\end{myitemize}

\subsubsection{判断和读取字符串的值}
\begin{myitemize}
	\item \mintinline{shell}{${var}} : 读取变量的值, 与 \mintinline{shell}{$var} 相同;
	
	\item \mintinline{shell}{${var:-DEFAULT}} : 若 \mintinline{shell}{var} 没有定义或为空时, 则返回 \mintinline{shell}{DEFAULT}, 否则返回变量的值;

	\item \mintinline{shell}{${var:=DEFAULT}} : 若 \mintinline{shell}{var} 没有定义或为空时, 则返回 \mintinline{shell}{DEFAULT}, 并将 \mintinline{shell}{var} 的值设置为 \mintinline{shell}{DEFAULT}, 否则返回变量的值;
	
	\item \mintinline{shell}{${var:+DEFAULT}} : 若变量已经赋值, 其值才会被替换为 \mintinline{shell}{DEFAULT}, 否则不进行任何替换;
	
	%\item \mintinline{shell}{var:?MESSAGE} : 
	
\end{myitemize}
	
\subsubsection{字符串替换}
\begin{myitemize}
	\item \mintinline{shell}{${var/SUBSTR/REPSTR}} : 使用 \mintinline{shell}{REPSTR} 来替换\textbf{第一个}匹配到的 \mintinline{shell}{SUBSTR};
	
	\item \mintinline{shell}{${var//SUBSTR/REPSTR}} : 使用 \mintinline{shell}{REPSTR} 来替换\textbf{所有}匹配到的 \mintinline{shell}{SUBSTR};
	
	\item \mintinline{shell}{${var/#SUBSTR/REPSTR}} : 如果 \mintinline{shell}{var} 的前缀匹配 \mintinline{shell}{SUBSTR}, 则用 \mintinline{shell}{REPSTR} 替换\textbf{匹配到的前缀} \mintinline{shell}{SUBSTR};
	
	\item \mintinline{shell}{${var/%SUBSTR/REPSTR}} : 如果 \mintinline{shell}{var} 的后缀匹配 \mintinline{shell}{SUBSTR}, 则用 \mintinline{shell}{REPSTR} 来替换\textbf{匹配到的后缀} \mintinline{shell}{SUBSTR}; 
			
\end{myitemize}

参考: \href{https://blog.csdn.net/qq_51470638/article/details/125035162}{SHELL字符串处理技巧}, \href{https://www.cnblogs.com/gaochsh/p/6901809.html}{linux shell 字符串操作详解}.
	
\subsection{不知道如何分类}
这里就暂时放一些我不知道该如何分类的知识点吧.

