\subsection{变量}
\paragraph{基本概念}
定义 shell 变量时不需要加 \$ 符号, 如 \mintinline{shell}{var=444}, 切记定义变量是\textbf{等号之间不能有空格}, 因为 shell 中将空格看作参数的分隔符, 则会把变量名当作函数名, 把剩下的部分当作参数传进去!

要使用一个定义过的变量时, 只要在变量名前加美元符号即可: \mintinline{shell}{$var}, 或者 \mintinline{shell}{${var}}. 变量名为的花括号是可选的, 加花括号是为了帮助 shell 解释器识别变量的边界, 例如 \mintinline{shell}{"This is ${var}-laboratory!"}

已经定义过的变量是可以重新定义的, 注意定义的时候是不加美元符号的!

删除变量: \mintinline{shell}{unset var}.

\paragraph{变量类型}
\begin{myitemize}
	\item 局部变量. 局部变量在脚本或命令中定义, 仅在当前shell实例中有效, 其他shell启动的程序不能访问局部变量;
	\item 环境变量. 所有的程序, 包括shell启动的程序, 都能访问环境变量, 有些程序需要环境变量来保证其正常运行. 必要的时候shell脚本也可以定义环境变量;
	\item shell变量. shell变量是由shell程序设置的特殊变量. shell变量中有一部分是环境变量, 有一部分是局部变量, 这些变量保证了shell的正常运行.
\end{myitemize}

\paragraph{只读变量}
使用 \mintinline{shell}{readonly} 命令可以将变量定义为只读变量: \mintinline{shell}{readonly var}.

\subsection{数组}
\paragraph{数组定义}
shell 中, 用括号表示数组, 元素之间用空格分开, 如: \mintinline{shell}{arr=(e1 e2 e3 e4 e5)}, 也可以每行写一个元素:

\begin{tcblisting}{listing engine=minted,minted style=trac,minted language=shell,listing only}
	# 写成一行
	arr=(e1 e2 e3 e4 e5)
	# 一行一个元素
	arr=(
	e1
	e2
	e3
	e4
	e5
	)
	# 或者直接单独定义数组的各个元素
	arr[0]=e1
	arr[1]=e2
	arr[n]=en
\end{tcblisting}

注意, 直接定义数组的各个元素时, 可以不定义某些索引处的值. 

\paragraph{读取数组}
一般的格式:  \mintinline{shell}{${var[i]}}, 或者取所有元素: \mintinline{shell}{${var[@]}}.

获取数组的长度: \mintinline{shell}{${var[@]}}, 或者 \mintinline{shell}{${var[*]}}. 注意, 计算长度时会忽略那些没有定义值的下标.


\subsection{注释}
单行注释: \mintinline{shell}{#}.
多行注释: 
\begin{tcblisting}{listing engine=minted,minted style=trac,minted language=shell,listing only}
	:<<EOF
	第一行注释
	第二行注释
	第三行注释
	EOF
\end{tcblisting}

\subsection{字符串}
\subsubsection{引号}
字符串可以用单引号, 可以用双引号, 也可以不用引号. 但是单引号和双引号还是有些差别的. 单引号的限制:
\begin{myitemize}
	\item 单引号里的任何字符都会原样输出, 里面的变量是无效的;
	
	\item 单引号中不能出现单独一个的单引号, 对单引号进行转义也不行, 但可以成对出现, 此时起着拼接字符串的功能.
\end{myitemize}

双引号的特点:
\begin{myitemize}
	\item 双引号离可以有变量;
	
	\item 双引号里可以出现转义字符.
\end{myitemize}

还有一种特殊的引号, 反引号: \mintinline{shell}{`}. 用反引号括起来的字符串会被 shell 当作命令. 在执行时, shell 首先会执行该命令, 并以它的标准输出结果替换整个反引号部分. 例如: \mintinline{shell}{echo "current working directory is `pwd`"}, 将输出: \mintinline{shell}{current working directory is /home/gzy}. 


\subsubsection{\#*, \#\#*, \%*, \%\%* 的含义及用法}
通过一个示例来解释, 定义一个变量: \mintinline{shell}{file=/dir1/dir2/dir3/my.file.txt}.

\begin{myitemize}
	\item \mintinline{shell}{${file#*/}} : 删除\textbf{第一个} \mintinline{shell}{/} 及其左边的字符串, 故该命令的值为 \mintinline{shell}{dir1/dir2/dir3/my.file.txt};
	
	\item \mintinline{shell}{${file##*/}} : 删除\textbf{最后一个} \mintinline{shell}{/} 及其左边的字符串, 故该命令的值为 \mintinline{shell}{my.file.txt};
	
	\item \mintinline{shell}{${file%/*}} : 删除\textbf{最后一个}(即\textbf{从右往左的第一个}) \mintinline{shell}{/} 及其右边的字符串, 故该命令的值为 \mintinline{shell}{/dir1/dir2/dir3};
			
	\item \mintinline{shell}{${file%%/*}} : 删除\textbf{第一个}(即\textbf{从右往左的最后一个}) \mintinline{shell}{/} 及其右边的字符串, 故该命令的值为 \mintinline{shell}{}(空值);
	
\end{myitemize}
注意, 上述的 \mintinline{shell}{/} 可以是其他字符. 显然, \mintinline{shell}{#} 是删除左边, \mintinline{shell} 表示从右向左匹配, \mintinline{shell}{#} 表示从左向右匹配, 都是非贪婪匹配, 即匹配到符合条件的最短结果. \mintinline{shell}{##}, \mintinline{shell} 则属于贪婪匹配, 即匹配符合条件的最长结果.
\end{myitemize}



\subsubsection{截取字符串}
定义一个字符串: \mintinline{shell}{var=12345678}.

\begin{myitemize}
	\item \mintinline{shell}{${var:OFFSET}}, 从 OFFSET 开始截取字符串. 如 \mintinline{shell}{${var:2}}, 即从下标 2 开始截取字符串 (shell 中下标也是从 0 开始), 结果为 \mintinline{shell}{345678};
	
	\item \mintinline{shell}{${var:OFFSET:LENGTH}}, 从 OFFSET 开始截取长度为 LENGTH 的子串. 如 \mintinline{shell}{${var:2:2}}, 即从下标 2 开始截取长度为 2 的串, 结果为 \mintinline{shell}{34};
	
	\item \mintinline{shell}{${var:0-OFFSET}}, 从导数第 OFFSET 个字符串开始截取串, 类似于 python 中的负的索引. 如 \mintinline{shell}{${var:0-2}}, 即从倒数第 2 个开始截取字符串, 结果为 \mintinline{shell}{78}.  那能不能不写 0 呢? 当然可以, 把 0 换成空格即可, 如 \mintinline{shell}{${var: -OFFSET}}; 或者这样 \mintinline{shell}{${var:(-OFFSET)}};
	
	\item \mintinline{shell}{${var:0-OFFSET:LENGTH}}, 从导数第 OFFSET 个字符串开始截取长度为 LENGTH 的子串. 如 \mintinline{shell}{${var:0-2:1}}, 即从倒数第 2 个字符开始截取长度为 1 的子串. 结果为 \mintinline{shell}{7};
\end{myitemize}

\subsubsection{判断和读取字符串的值}
\begin{myitemize}
	\item \mintinline{shell}{${var}} : 读取变量的值, 与 \mintinline{shell}{$var} 相同;
	
	\item \mintinline{shell}{${var:-DEFAULT}} : 若 \mintinline{shell}{var} 没有定义或为空时, 则返回 \mintinline{shell}{DEFAULT}, 否则返回变量的值;

	\item \mintinline{shell}{${var:=DEFAULT}} : 若 \mintinline{shell}{var} 没有定义或为空时, 则返回 \mintinline{shell}{DEFAULT}, 并将 \mintinline{shell}{var} 的值设置为 \mintinline{shell}{DEFAULT}, 否则返回变量的值;
	
	\item \mintinline{shell}{${var:+DEFAULT}} : 若变量已经赋值, 其值才会被替换为 \mintinline{shell}{DEFAULT}, 否则不进行任何替换;
	
	%\item \mintinline{shell}{var:?MESSAGE} : 
	
\end{myitemize}
	
\subsubsection{字符串替换}
\begin{myitemize}
	\item \mintinline{shell}{${var/SUBSTR/REPSTR}} : 使用 \mintinline{shell}{REPSTR} 来替换\textbf{第一个}匹配到的 \mintinline{shell}{SUBSTR};
	
	\item \mintinline{shell}{${var//SUBSTR/REPSTR}} : 使用 \mintinline{shell}{REPSTR} 来替换\textbf{所有}匹配到的 \mintinline{shell}{SUBSTR};
	
	\item \mintinline{shell}{${var/#SUBSTR/REPSTR}} : 如果 \mintinline{shell}{var} 的前缀匹配 \mintinline{shell}{SUBSTR}, 则用 \mintinline{shell}{REPSTR} 替换\textbf{匹配到的前缀} \mintinline{shell}{SUBSTR};
	
	\item \mintinline{shell}{${var/%SUBSTR/REPSTR}} : 如果 \mintinline{shell}{var} 的后缀匹配 \mintinline{shell}{SUBSTR}, 则用 \mintinline{shell}{REPSTR} 来替换\textbf{匹配到的后缀} \mintinline{shell}{SUBSTR}; 
			
\end{myitemize}

参考: \href{https://blog.csdn.net/qq_51470638/article/details/125035162}{SHELL字符串处理技巧}, \href{https://www.cnblogs.com/gaochsh/p/6901809.html}{linux shell 字符串操作详解}.

\subsection{awk}
早有耳闻 \mintinline{shell}{awk} 的大名, 虽然之前有了解过但是一直觉得很麻烦就没好好学, 导致现在想要用 \mintinline{shell}{awk} 统计文件的列数都不会. 悲哀, 真是悲哀!

\subsubsection{awk 简介}


\subsection{vim}
\subsubsection{常用操作}
\begin{myitemize}
	\item 跳转到文件尾部: 命令行模式下输入 \mintinline{shell}{$};
\end{myitemize}

\subsection{不知道如何分类}
这里就暂时放一些我不知道该如何分类的知识点吧.

