参考文档:\href{https://www.cplusplus.com/}{cplusplus.com}
\subsection{C++数据类型转换}
\paragraph{数字 <=> 字符串}
\begin{itemize}
	\item itoa: 整数转为字符串,可以设置基底
	\item stoi: 字符串转为整数 ,可以设置基底
	\item atoi、atol、atoll: 将字符串转为int/long/long long,可以设置基底
	\item to\_string: 可以讲数字转为字符串,包括整数、小数等
\end{itemize}

\subsection{C++集合} 通过$\#include<set>$来引入set。关于集合的一些运算,包含在algorithm库中。在algorithm库中有一个includes函数,可以用来判断两个容器之间是否存在包含关系,但是,要注意\tbc{red}{两个容器都要先从小到大进行排序!!!}
\begin{cpp}
	// 省略头文件和库的包含代码
	int container[] = {5,10,15,20,25,30,35,45,50};
	int continent[] = {40,30,20,10};
	
	sort (container,container+10);
	sort (continent,continent+4);
	
	// using default comparison:
	if ( std::includes(container,container+10,continent,continent+3) )
		cout << "container includes continent!\n";
	else	cout<<"NO\n";
	
\end{cpp}


\subsection{字符串}
C++中字符串用双引号括起来,字符用单引号括起来。
\paragraph{字符串比较}
compare

\paragraph{常用方法}
\begin{itemize}
	\item 初始化:\mintinline{cpp}{string s(n, 'c');}
	\item 插入字符串:\mintinline{cpp}{s.insert(pos, len, "insert");}
	\item 删除子串:\mintinline{cpp}{s.erase(pos, len, "erase");}
	\item 
\end{itemize}

\subsection{数组}
\paragraph{数组初始化}
\begin{itemize}
	\item \mintinline{cpp}{type [][depth][depth]...[depth]}声明多维数组时,只能有一维的大小不指定
	\item \mintinline{cpp}{int arr[5] = {1, 2, 3, 4, 5}}声明数组时可以给数组赋值(初始化列表),如果指定的数组大小则$\{\}$中的值的数量应该小于等于数组大小,如果数量小于数组大小,则数组中后续位置将会被填充默认值(对于基础类型,则为0)。因此,当$\{\}$未空时,数组将会被填充为默认值
	\item \mintinline{cpp}{int *p = new int[5]}将在堆中分配内存,此时也可以对数组进行初始化:\mintinline{cpp}{int *p = new int[]{1, 2, 3, 4, 5}}、\mintinline{cpp}{int *p = new int[]()}。对于基本类型,如果不进行初始化,局部定义的数组的值是未定义的;对于类类型,会默认调用其默认构造函数。\textbf{堆中分配的数组要用delete删除}
\end{itemize}

\subsection{遍历vector的方式}
\begin{cpp}
	vector<int> nums = {1, 2, 3, 4, 5};
	// 方法1:下标
	for(int i = 0; i < nums.size(); i++){
		// do something
	}
	
	// 方法2:迭代器
	for(vector<int>::iterator it = nums.begin(); it != nums.end(); it++){
		// do something
	}

	// 方法3:auto,以引用的形式遍历,可以修改值
	for(auto &it: nums){
		// do something 
	}

	// 方法4:auto,以值的形式遍历,不可以修改值
	for(auto it: nums){
		// do something
	}
	
	// 方法5:for_each
	foreach(nums.begin(), nums.end(),
			[](const auto &val) -> void { 
				// do something
			})
	
\end{cpp}

\subsection{for\_each}
for\_each源码如下:
\begin{cpp}
	template<class InputIterator, class Function>
	Function for_each(InputIterator first, InputIterator last, Function fn)
	{
		while (first!=last) {
			fn (*first);
			++first;
		}
		return fn;      // or, since C++11: return move(fn);
	}
\end{cpp}
其中参数fn是一个一元函数(即只能有一个参数),它的返回值会被忽略。

\subsection{deque}
\mintinline{cpp}{#include <deque>}。double ended queue,双端队列。\href{https://www.cplusplus.com/reference/deque/deque/?kw=deque}{dequ}e是一种动态容器(通常以动态数组的形式,如链表,来实现),可以在两端增加或者删除元素,随机访问deque中的元素。

\subsection{priority\_queue}
\mintinline[bgcolor=yellow]{cpp}{#include <queue>}。
\begin{minted}{cpp}
	template <class T, class Container = vector<T>,
	class Compare = less<typename Container::value_type> > class priority_queue;
\end{minted}
\href{https://www.cplusplus.com/reference/queue/priority_queue/?kw=priority_queue}{priority\_queue},优先队列。priority\_queue是一种容器适配器(container adaptor),它使用某种容器(可以是queue、vector等,默认是vector)作为其数据的容器,并提供了一些接口来访问容器内的元素,重点就在这些接口上(\mintinline{cpp}{push, pop, top})!priority\_queue的特点:
\begin{myitemize}
	\item 第一个元素是优先级最大/小的元素(大/小根堆)
	\item \mintinline{cpp}{top} :返回顶部的元素,即优先级最大/最小的元素
	\item \mintinline{cpp}{pop} :删除顶部的元素
\end{myitemize}
例子:
\begin{minted}[linenos, breakafter=true,autogobble=true,frame=leftline]{cpp}
	#incldue <queue>
	
	int main(){
		// 大根堆
		priority_queue<int> a; // <=> priority_queue<int, vector<int>, less<int> >a
		
		// 小根堆
		priority_queue<int, vector<int>, greater<int> > b;
	}
\end{minted}


