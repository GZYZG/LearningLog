\subsection{pytorch tensor.view}
\href{https://pytorch.org/docs/stable/tensor_view.html}{view()}相当于数据库中的view --- 对数据进行查看,是查看数据的一种方式。使用view()不会产生数据的复制,与原tensor共享同一块内存,修改view会使原tensor发生变化。

\subsection{nn.Dropout}
\mintinline{python}{torch.nn.Dropout(p=0.5, inplace=False)}。\mintinline{python}{nn.Dropout}实现了Dropout,可以作为神经网络中的层来使用。对于输入\mintinline{python}{nn.Dropout}的数据,会以参数为$p$的伯努利分布对每个channel的数据进行采样然后置零(每个channel中的每个元素有$p$的概率置零)。\textbf{注意:}在训练时,\mintinline{python}{nn.Dropout}的输出会乘以$\frac{1}{1 - p}$进行缩放,不训练时,则等于恒等映射。不管是否有元素被置零,都会乘以$\frac{1}{1 - p}$,$p$越大,放大的倍数越大。