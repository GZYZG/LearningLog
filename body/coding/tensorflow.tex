\paragraph{tf.variable\_scope \& tf.name\_scope}

\paragraph{tf.sparse\_tensor}	tf中的稀疏张量。使用三个稠密的张量来表示。
\begin{itemize}
	\item indices: 表示原张量中非零值的位置
	\item values: 表示indices中元素所指位置上的值
	\item dense\_shape: 表示原张量的shape
\end{itemize}

\paragraph{tf.train.Saver}tf中用于保存、恢复模型参数的接口。主要由两个接口:1)tf.train.Saver().save()用于保存模型;2)tf.train.Saver().restore()用于回复模型。
\begin{python}
	import tensorflow as tf
	saver = tf.train.Saver(max\_to\_keep=3) # max\_to\_keep表示保存的checkpoint最大次数
	...
	# 保存模型
	# sess: 会话的名字
	# save\_path: 模型的保存路径
	# global\_step: 保存模型时的后缀
	# 使用以下方法保存模型后会产生四个文件,分别是:
	# checkpoint文件:会记录最新的模型是哪个
	# .ckpt.meta文件:包含元图,保存了计算图的结构,没有变量的值
	# .ckpt.data 文件:保存权重等参数
	# .ckpt.index 文件:为数据文件提供索引,{还不太确定}
	saver.save(sess=sess, save_path=model_save_path, global_step=step)
	...
	# 恢复模型
	# save_path可以不用加模型的后缀
	saver.restore(sess=sess, save_path=model_save_path)
\end{python}

\paragraph{tf.Session}运行tf中操作的类,Session类封装了运行操作所需要的环境。可以使用手动开启和关闭、上下文的方式使用Session。在使用Session时还可以进行配置,将配置作为Session初始化的参数传入。

\paragraph{tf中的RNN}RNN的训练数据与其他神经网络的训练数据有一点不一样。通常的NN中的训练数据,每个样本就是一个向量。但是在RNN中,每个训练样本就是一个矩阵,每一行表示一步的输入,正因为RNN所处理的问题不同,RNN的输出数据也变得更加复杂。\\
针对不同的RNN有不同形式的输出,但可以进行简单的归纳:对于RNN中的每个样本,其形式为$\mathbb{R}^{max\_time \times embed\_size}$,其中$max\_time$表示所有样本中序列的最大长度,$embed\_size$表示每一步的输入的长度。当输入了一个样本后,会将样本的每一行 --- 即每一步的数据输入到RNN中,由于RNN本身的特点,每一步都可以产生输出,通常来说包含每一步的输出和状态(具体到不同的RNN时会有不同)。每一步的输出维度会收到RNN模型隐层宽度 --- 即隐层单元数量的影响。并且针对不同的任务,最终需要的输出是不一样,例如m对n、m对1、m对m的任务等等。\\
在tf中,有多种关于RNN单元的类,如BasicLSTMCell、GRUCell、MultiRNNCell。可以在这些类的基础上搭建自己的RNN模型。


\paragraph{ImageDataGenerator}
这是Keras中的一个图像数据生成器,同时也可以在batch中对数据进行增强,扩充数据集大小,增强模型的泛化能力。比如进行旋转,变形,归一化等。

\paragraph{tf.where}
tf.where(condition, x=None, y=None, name=None)

condition是一个布尔数组,condition、x、y的维度必须相同。

其中x、y可以为None,此时返回的是一个二维数组,该二维数组的行数表示condition中为True的元素的数量,每一行代表对应为True元素的坐标。

当x、y不为None是(x,y必须同时不为None),则会创建一个与condition同样shape的张量,该张量某位置的元素的取值来自x或y,当该位置在condition中为True是则来自x,否则来自y。

\paragraph{pytorch tensor.view}
\href{https://pytorch.org/docs/stable/tensor_view.html}{view()}相当于数据库中的view --- 对数据进行查看,是查看数据的一种方式。使用view()不会产生数据的复制,与原tensor共享同一块内存,修改view会使原tensor发生变化。

\paragraph{TFRecord}
下列表来自\href{https://www.jianshu.com/p/72596a8488c3}{简书}:
\begin{itemize}
	\item Record顾名思义主要是为了记录数据的。
	\item 为了更加方便的建图,原来使用placeholder的话,还要每次feed\_dict一下,使用TFRecord+ Dataset 的时候直接就把数据读入操作当成一个图中的节点,就不用每次都feed了。
	\item 可以方便的和Estimator进行对接。
	\item TFRecord以字典的方式进行数据的创建。
\end{itemize}

\paragraph{tf.Graph}
不同Graph之间的变量是不可以共享的!


\paragraph{tf.pad}
这是tf种对tensor进行填充的函数。\textit{tf.pad(tensor, paddings, mode='CONSTANT', constant\_values=0, name=None)}。其中tensor为要被填充的张量;paddings表示填充时要填充多少值进去,控制了填充后tensor的大小;mode表示填充的模式,什么样的值被填充进去;constant\_value表示mode为CONSTANT时的常量值。其中最重要的参数是\textit{paddings}。\textit{paddings}是一个形状为[n,2]的tensor,n为input的rank(其实可以简单理解为len(input.shape))。\textit{paddings}中的第$i$个元素可以看做是一个长度为2的数组,第一个元素表示了在input的第$i$维的前面增加的大小,第二个元素表示在第$i$维的后面增加的大小,那么对于input的第$i$来说,它的大小较原来增大了$paddings[i][0]+paddings[i][1]$。

\paragraph{tf.tile}平铺张量。\textit{tf.tile(input, multiples, name=None)}。其中input是一个张量;multiples是一个1-D数组,数组的长度是input的维度数量,每个元素的值表示input被复制的次数,即\textit{input.dims[i] = input.dims[i] * multiples[i]}。

\paragraph{tf.squeeze}压缩张量,将张量中维度大小为1维压缩掉。

\textit{tf.squeeze(input, axis=None, name=None)}。

input为输入的张量;axis为可选参数,可以为整数或者整数1-D数组,为None时表示移除所有大小为1的维,否则移除指定的维。

反之则为$unsqueeze$。

\paragraph{tf.expand_dims}给数据增加维度。\textit{tf.expand\_dims(input, axis, name)},表示在$input$的第$axis$维插入一个维度,维度大小为1。
