% 记录日常的一些想法
\paragraph{基于深度神经网络的模型也可以看作是一种图,能否使用gml的一些方法对其进行分析?} 很可惜,这个想法已经被发表成论文了\cite{you2020graph}。

但是这方面地工作还比较少,可以做的方向有:
\begin{itemize}
	\item 如何对Deep neural network进行建模,现在地DL模型中,有着众多的变化,如网络中的DROP技术、pooling、随着训练单元之间的连接权重变化等
	\item {\color{red}神经网络模型的训练会随着训练发生变化,能否将其视作动态图?}
	\item 对深度网络建模后,对其进行分析,分析其具有何种性质,与现有的网络常识、规律进行比较
	\item 能够通过对深度网络的分析,得到神经网络所蕴含的更高层的语义,对模型进行解释?
	\item 能否通过对深度神经网络的分析对深度模型进行改善?
\end{itemize}

\paragraph{生物的大脑中的神经元之间互相联系,也可以看作一种图,能否以研究图的一些方法对大脑的神经系统进行研究?} 已经有相关的论文了!!!
\paragraph{将代码视作图}	已经有相关论文。数据集 LINUX \cite{6228085} 内核中的程序依赖图,每个图代表一个函数,每个结点代表一条语句,边表示语句之间的依赖。
\paragraph{找到科研新点子的方法} 纵览研究领域,看看传统方法都是怎么解决领域内的主要问题的,或者解决了一些什么问题,再看看传统方法的局限性,或者如何用新的方法解决领域内的问题。
\paragraph{图领域一些难点,未解决的问题}\newline
1. 影响力最大化问题
2. 影响力问题中对negative influence的考虑


\paragraph{武林外传(my own swordsman}\\
1. 吃再多苦,只当自己是二百五;受再多罪,只当自己是窝囊废 
2. 只要给够加班费,当牛做马无所谓

\paragraph{神经网络中的一些点子}\\1. 设计新的神经单元
2. 设计新的网络结构
3. 开发更有效的正则化策略

\paragraph{从图的表征中重建图} 是否能够从图的表征或者图中所有结点的表征为基础来重建传统的图的结构$G = (V, E) $。这样的重建可以作为图数据的一种加密方式、图数据的压缩保存等。GraphRNN\cite{you2018graphrnn}中的图生成,反过来是否能看作从表征重构图的过程?{\color{red}更准确的讲,这应该叫图地编码解码!}

对于图的编码解码,已经有相关的工作\cite{simonovsky2018graphvae},但这方面的工作还比较少。

\paragraph{自动文献调研} 给定一个领域限定,能够自动地调研相关领域地资料(文献,网路资源),能够在调研后发现问题新的解决方法,或者挖掘出新的问题。那能否完成\textbf{自动写论文呢}\cite{wang-etal-2019-paperrobot}?(自动文档摘要可以作为这样的一个系统的一部分)

\paragraph{基于多层次的图相似性计算}从输入的图,生成每个图的多层级表示,在多个层级的表示上进行相似度计算。这方面现在已经有一些相关的工作,不过还比较少。多个层级的划分可以基于以下的标准:
\begin{itemize}
	\item 可以基于不同的表示粒度,如结点、超点、图的表征
	\item 使用不同的方法来获得不同层次的表示
	%\item 
\end{itemize}
在计算图的相似度时主要有这么几个难点:
\begin{itemize}
	\item 图的同构性问题。一个图的结点排列是不确定的,基于不同的结点排列可能会有不同的表示
	\item 要是inductive的。能够泛化到未见过的图
	\item 能对大图进行相似度的计算。由于图的规模一大,对其进行比较将会是一件耗时的操作
	\item {\color{red}图数据的丰富性}。例如多重图、有向图、异构图、知识图谱等
	\item {\color{red}动态图}。其实,目前大部分工作是在静态图上开展的,如何现有研究工作扩展到动态图还有很大的研究空间
\end{itemize}


\paragraph{在结点表征的基础上做一些延申性的工作}比如链接预测、动态的链接预测等。
g
\paragraph{针对图数据库,基于图相似性的图表征}在一个很大的图数据库中,如果我们得到了任意两个图之间的相似性,能否利用相似性来计算图的表征呢?这个想法源于:基于邻居信息聚合的结点的表征是利用结点之间的关系(可以认为是相似性)来聚合邻居结点信息。那么,{\color{red}在一个很大的图数据库中,能否构造一个超图,利用图之间的相似性构造一个这样的超图呢?,每一个图数据作为超图中的点,边可以以图之间的相似性来构建。在这样的一个超图的基础上来学习图的表征?}在这个想法中,关键之处在于相似性如何得到,现有的一些图相似性的计算就是基于图的表征来计算的,那这样不就是多此一举了吗?

\paragraph{GNN生成的表征空间的问题}对于使用不同的GNN/GCN生成的结点/图表征,它们之间存在什么样的关系呢?是否会受到数据本身的影响呢?比如一个GNN将一个图中的所有结点映射到一个向量空间中,另一个GNN把所有结点映射到另一个向量空间中,这两个空间会存在某种关系吗,存在什么样的关系呢?{\textbf{\color{red}这种关系是否反映了不同GNN模型之间的关系呢?}}
