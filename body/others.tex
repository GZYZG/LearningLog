\subsection{李沐 --- 用随机梯度下降来优化人生}
\textbf{要有目标}. 你需要有目标. 短的也好, 长的也好. 认真定下来的也好, 别人那里捡来的也好. 就跟随机梯度下降需要有个目标函数一样. 

\textbf{目标要大}. 不管是人生目标还是目标函数, 你最好不要知道最后可以走到哪里. 如果你知道, 那么你的目标就太简单了, 可能是个凸函数. 你可以在一开始的时候给自己一些小目标, 例如期末考个80分, 训练一个线性模型. 但接下来得有个更大的目标, 财富自由也好, 100亿参数的变形金刚也好, 得足够一颗赛艇. 

\textbf{坚持走}. 不管你的目标多复杂, 随机梯度下降都是最简单的. 每一次你找一个大概还行的方向(梯度), 然后迈一步(下降). 两个核心要素是方向和步子的长短. 但最重要的是你得一直走下去, 能多走几步就多走几步. 

\textbf{痛苦的卷}. 每一步里你都在试图改变你自己或者你的模型参数. 改变带来的痛苦. 但没有改变就没有进步. 你过得很痛苦不代表在朝着目标走, 因为你可能走反了. 但是过得很舒服那一定在原地踏步. 需要时刻跟自己作对. 

\textbf{可以躺平}. 你用你内心的激情来迈步子. 步子太小走不动, 步子太长容易过早消耗了激情. 周期性的调大调小步长效果挺好. 所以你可以时不时休息休息. 

\textbf{四处看看}. 每一步走的方向是你对世界的认识. 如果你探索的世界不怎么变化, 那么要么你的目标太简单, 要么你困在你的舒适区了. 随机梯度下降的第一个词是随机, 就是你需要四处走走, 看过很多地方, 做些错误的决定, 这样你可以在前期迈过一些不是很好的舒适区. 

\textbf{快也是慢}. 你没有必要特意去追求找到最好的方向和最合适的步子. 你身边当然会有幸运之子, 他们每一步都在别人前面. 但经验告诉我们, 随机梯度下降前期进度太快, 后期可能乏力. 就是说你过早的找到一个舒适区, 忘了世界有多大. 所以你不要太急, 前面徘徊一段时间不是坏事. 成名无需太早. 

\textbf{赢在起点}. 起点当然很重要. 如果你在终点附近起步, 可以少走很多路. 而且终点附近的路都比较平, 走着舒服. 当你发现别人不如你的时候, 看看自己站在哪里. 可能你就是运气很好, 赢在了起跑线. 如果你跟别人在同一起跑线, 不见得你能做更好. 

\textbf{很远也能到达}. 如果你是在随机起点, 那么做好准备面前的路会非常不平坦. 越远离终点, 越人迹罕至. 四处都是悬崖. 但随机梯度下降告诉我们, 不管起点在哪里, 最后得到的解都差不多. 当然这个前提是你得一直按照梯度方向走下去. 如果中间梯度炸掉了, 那么你随机一个起点, 调整步子节奏, 重新来. 

\textbf{独一无二}. 也许大家有着差不多的目标, 在差不多的时间毕业买房结婚生娃. 但每一步里, 每个人的内心中看到的世界都不一样, 导致走的路不一样. 你如果跑多次随机梯度下降, 在各个时间点的目标函数值可能都差不多, 但每次的参数千差万别. 不会有人关心你每次训练出来的模型里面参数具体是什么值, 除了你自己. 

\textbf{简单最好}. 当然有比随机梯度下降更复杂的算法. 他们想每一步看向更远更准, 想步子迈最大. 但如果你的目标很复杂, 简单的随机梯度下降反而效果最好. 深度学习里大家都用它. 关注前面, 每次抬头瞄一眼世界, 快速做个决定, 然后迈一小步. 小步快跑. 只要你有目标, 不要停, 就能到达. 
